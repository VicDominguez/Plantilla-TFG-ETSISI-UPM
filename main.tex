%Definimos la clase del documento y cargamos las librerías
\documentclass[a4paper]{report}
\usepackage{graphicx}
\usepackage[utf8]{inputenc}
%paquetes para la portada azul
\usepackage{afterpage}
\usepackage{xcolor}
%Para tener los elementos de texto en español
\usepackage[spanish]{babel}
%Hipervinculos para todo lo útil
\usepackage{hyperref}
\usepackage{fancyhdr}
%evitamos que rompa las palabras
\usepackage[none]{hyphenat}
%añadir ficheros externos
\usepackage{subfiles}
%soporte para anexos
\usepackage{appendix}
%importacion automatica de codigo en los anexos
\usepackage{minted}
%paquetes para figuras
\usepackage{caption}
\usepackage{subcaption}

%Encabezados bonitos
\pagestyle{headings}

%paquete para poner la bibliografia en la table of contents
\usepackage[nottoc,notlot,notlof]{tocbibind}

%Cambiamos el espacio en blanco enorme que por defecto hay en las páginas inicio de capítulo
\usepackage{titlesec}
\titleformat{\chapter}[display]   
{\normalfont\huge\bfseries}{\chaptertitlename\ \thechapter}{20pt}{\Huge}   
\titlespacing*{\chapter}{0pt}{-50pt}{40pt}

%para crear subsubsubsection
\setcounter{secnumdepth}{4}
\titleformat{\paragraph}
{\normalfont\normalsize\bfseries}{\theparagraph}{1em}{}
\titlespacing*{\paragraph}
{0pt}{3.25ex plus 1ex minus .2ex}{1.5ex plus .2ex}

%definimos el azul de la portada
\definecolor{AzulPortada}{RGB}{135, 206, 250}

%Ruta en la que pondremos las imágenes para el documento
\graphicspath{ {imagenes/} }
%Redifinimos el nombre que asigna babel spanish a las tablas (cuadro) a tabla
\renewcommand{\spanishtablename}{Tabla}
\renewcommand{\spanishlisttablename}{Índice de tablas}
\renewcommand{\spanishcontentsname}{Índice}
\renewcommand{\appendixname}{Anexos}
\renewcommand{\appendixtocname}{Anexos}
\renewcommand{\appendixpagename}{Anexos}

% Definimos el comando de página en blanco
\newcommand\paginablanco{%
    \null
    \thispagestyle{empty}%
    \newpage}

% Definimos el comando de página en blanco sin avanzar numeracion
\newcommand\paginablancosin{%
    \paginablanco{}
    \addtocounter{page}{-1}}

%mayor anchura en las tablas
\renewcommand{\arraystretch}{1.5}
%mayor ancuhura en las fracciones
\newcommand\ddfrac[2]{\frac{\displaystyle #1}{\displaystyle #2}}
%coamdno escribir TFG
\newcommand{\tfg}{Trabajo Fin de Grado }


% Comenzamos el documento
\begin{document}
% Hacemos la portada
\begin{titlepage}
\pagecolor{AzulPortada}\afterpage{\nopagecolor}

\begin{figure}[!htb]
   \begin{minipage}{0.45\textwidth}
     \centering
     \includegraphics[width=1\textwidth]{logo_upm.png}
   \end{minipage}\hfill
   \begin{minipage}{0.45\textwidth}
     \centering
     \includegraphics[width=1\textwidth]{logo_etsisi.png}
   \end{minipage}
\end{figure}

%{\includegraphics[width=0.5\textwidth]{logo_etsisi_bn.png}\par}
{\bfseries\Huge \textcolor{black}{Título del proyecto} \par}
\vfill
{\huge \textcolor{black}{Proyecto Fin de Grado} \par}
\vfill
{\huge \textcolor{black}{Grado en XXXX} \par}
\vfill
{\huge \textcolor{black}{Autor:} \par}
{\huge \textcolor{black}{Nombre Apellidos} \par}
\vfill
{\huge \textcolor{black}{Tutores:} \par}
{\huge \textcolor{black}{Nombre Apellidos} \par}
{\huge \textcolor{black}{Nombre Apellidos} \par}
\vfill
{\huge \textcolor{black}{Mes Año} \par}
\newpage

\thispagestyle{empty}
\centering % Para centrar la portada
% Para entender los siguientes comandos, consúltese los siguientes enlaces
% https://manualdelatex.com/tutoriales/crear-una-portada
% https://manualdelatex.com/tutoriales/tipo-de-letra
{\scshape\Large  Universidad Politécnica de Madrid \par}
{\scshape\Large Escuela Técnica Superior de Ingeniería de Sistemas Informáticos \par}
\vfill
{\includegraphics[width=1\textwidth]{logo_etsisi.png}\par}
\vfill
{\bfseries\LARGE Título del proyecto \par}
\vfill
{\Large Proyecto Fin de Grado \par}
\vfill
{\Large Grado en XXXX \par}
\vfill
{\Large Curso académico xxxx-xxxx \par}
\vfill
{\Large Autor: \par}
{\Large Nombre Apellidos \par}
\vfill
{\Large Tutores: \par}
{\Large Nombre Apellidos \par}
{\Large Nombre Apellidos \par}

\end{titlepage}

\pagenumbering{Roman} % para comenzar la numeracion de paginas en numeros romanos

% Agradecimientos
\chapter*{}
\thispagestyle{empty}
\addcontentsline{toc}{section}{Agradecimientos} % si queremos que aparezca en el índice
\begin{flushright}
\textit{Tus agradecimientos aqui}
\end{flushright}


\chapter*{Resumen} % si no queremos que añada la palabra "Capitulo"
\addcontentsline{toc}{section}{Resumen} % si queremos que aparezca en el índice
Por realizar

\chapter*{Abstract} % si no queremos que añada la palabra "Capitulo"
\addcontentsline{toc}{section}{Abstract} % si queremos que aparezca en el índice
Por realizar

\tableofcontents
\newpage
\listoftables %ver si hay alguna tabla al final
\newpage
\listoffigures
\newpage

\paginablanco{}

\chapter{Introducción}
\pagenumbering{arabic}
\subfile{secciones/introduccion}
\chapter{Estado del arte}
\subfile{secciones/estado_arte}
\chapter{Desarrollo del proyecto}
\subfile{secciones/desarrollo}
\chapter{Resultados}
\subfile{secciones/resultados}
\chapter{Conclusiones y trabajos futuros}
\subfile{secciones/conclusiones}
\newpage
\subfile{secciones/referencias}

\appendix
\clearpage
\addappheadtotoc
\appendixpage

\chapter{Código fuente del proyecto}
\subfile{anexos/codigo}

\end{document}